\documentclass{resume}
\RequirePackage{resume}

\begin{document}
	\begin{center}
		\name{何林璇}
		178-1211-0260 | hlx95108@gmail.com\\
		https://github.com/h9tyf\\
	\end{center}

	\section{教育经历}
	\datedsubsection{{\bfseries 北京航空航天大学} - 计算机科学与技术 - 本科}{2018年09月 - 2022年07月}
	\datedsubsection{{\bfseries GPA} : 3.61/4.00}{}
	\datedsubsection{{\bfseries 相关课程} : 数据结构(93)、计算机组成(91)、操作系统(89)、编译技术(82)、
	面向对象设计与构造(97)、智能计算体系结构(94)、数据挖掘(90)、Ruby程序设计(96)、Swift程序设计(100)。}{}


	\section{专业技能}
		\datedsubsection{{\bfseries 技能}:Python,Java,SQL,C,C++,C\#,Swift,Ruby,HTML,MATLAB,Lingo}{}
		
		\begin{itemize}
		    \item 对Python、Java、C、C++比较熟悉,用来完成过多个项目。
		    \item 关于C\#、Swift、Ruby,做过相关项目,了解有限。
		    \item 其他语言例如HTML,仅在需要使用时有了解一些。
		\end{itemize}
		
		
		%\datedsubsection{{\bfseries 语言}:英语(CET-4),英语(CET-6)}{}

	\section{开源项目及作品}
	以下项目均可以在我的GitHub个人主页上找到。
	\datedsubsection{{\bfseries 多线程单元作业评测机} - Java。为了对自己面向对象多线程单元的作业进行评测,与同学共同完成的一个评测机。主要有运行java代码并获取输出、对输出进行评测两个部分。我完成了对结果的检测部分。}{}
		
	\datedsubsection{{\bfseries 简易电子商务系统} - Ruby、HTML。利用Ruby on Rails框架实现,可供用户进行销售与购物。
	项目已经部署在Heroku,地址为https://sleepy-plains-42331.herokuapp.com/}{}
	
	\datedsubsection{{\bfseries 2020年数学建模国赛C题} - Python。 通过分析企业的进销项发票,从销方/购方单位代号,金额,税额,发票状态等计算财务指标和非财务指标,建立风险评估模型的基础数据来源,从而评估中小微企业必要的财务指标信息,我负责了其中全部代码的编写。
    这次比赛我们获得了北京市甲组一等奖。}{}
   
	\datedsubsection{{\bfseries 类C语法编译器} - C++。用 C++ 编写的简单编译器,使用自动机进行词法分析,使用递归下降方法进行语法分析,可以实现类 C 语法程序到 MIPS 代码的编译。编译技术课程的大作业项目。}{}
	
	\datedsubsection{{\bfseries Serverless工作负载预测} - Python。参加DataFountain的“大数据时代的Serverless工作负载预测”比赛的代码。对已有数据进行特征提取和处理,利用LightGBM框架对工作负载进行预测,方便Serverless软件架构可根据业务工作负载进行弹性资源调整。此次比赛最终名次为42/364。}{}
	
	\datedsubsection{{\bfseries 音频处理与识别} - Python。通过傅里叶变换对已有音频进行处理,并划分数据集,搭建CNN网络进行训练,并对新的音频进行识别,可以以90\%左右的精确率识别音频中声音的种类。}{}
	
	\datedsubsection{{\bfseries 简易超市管理系统} - C\#。简单的超市管理与购物系统。实现了管理员对商品的管理、以及对消费记录的查看功能,顾客的购物结算功能。同时设计有图形界面。}{}
	
	\section{实习经历}
	我参与了学院毕业设计提交与答辩的平台开发。
	这个项目使用了AWS PaaS平台来实现大多数功能,我主要负责在现有功能无法满足需求时
	利用接口编程来实现新的功能。例如在将学生提交的文件转发给对应老师时,平台无法实现自动转发,我通过Java对数据库的操作以及平台相关接口实现了这一功能。
	目前平台准备进行最后测试,并预计会在今年实现投入使用。从我们学院21届毕业生开始,都会使用这个平台进行毕业设计周报的提交以及答辩的准备。

\end{document}